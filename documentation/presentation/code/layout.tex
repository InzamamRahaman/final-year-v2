
\centering

\tikzstyle{switch}=[circle,
                                                thick,
                                                minimum size=0.9cm,
                                                draw=orange!80,
                                                fill=orange!25]

% The control input vector is represented by a purple circle.
\tikzstyle{input}=[rectangle,
                                    thick,
                                    minimum size=2.5cm,
                                    draw=purple!80,
                                    fill=purple!20]
                                    
\tikzstyle{background}=[rectangle,
                                                fill=gray!10,
                                                inner sep=0.2cm,
                                                rounded corners=5mm]

% The input, state transition, and measurement matrices
% are represented by gray squares.
% They have a smaller minimal size for aesthetic reasons.
\tikzstyle{matrx}=[rectangle,
                                    thick,
                                    minimum size=1cm,
                                    draw=gray!80,
                                    fill=gray!20]


\begin{tikzpicture}[>=latex,text height=1.5ex,text depth=0.25ex]
    % "text height" and "text depth" are required to vertically
    % align the labels with and without indices.
  
  % The various elements are conveniently placed using a matrix:
  \matrix[row sep=0.5cm,column sep=0.5cm] {
    % First line: Control input
    %&
        \node (lc1) [input]{logic cell}; &
        \node (s1)  [switch]{switch}; &
        \node (lc2) [input]{logic cell}; &
        \node (s2)  [switch]{switch}; &
        \node (lc3) [input]{logic cell}; &
        
        \\ 
        \node (s3) [switch]{switch}; &
        \node (s4) [switch]{switch}; &
        \node (s5)   [switch]{switch}; &
        \node (s6)   [switch]{switch}; &
        \node (s7) [switch]{switch}; &
        \\
         
        \node (lc4) [input]{logic cell}; &
        \node (s8)  [switch]{switch}; &
        \node (lc5) [input]{logic cell}; &
        \node (s9)  [switch]{switch}; &
        \node (lc6) [input]{logic cell}; &
        \\
    };
    
    % The diagram elements are now connected through arrows:
    \path[=]
        (lc1) edge[thick] (s1)
        (s1)  edge[thick] (lc2)
        (lc2) edge[thick] (s2)
        (s2) edge[thick] (lc3)
        (lc1) edge[thick] (s3)
        (s1) edge[thick] (s4)
        (lc2) edge[thick] (s5)
        (s2) edge[thick] (s6)
        (lc3) edge[thick] (s7)
        (s3) edge[thick] (s4)
        (s4) edge[thick] (s5)
        (s5) edge[thick] (s6)
        (s6) edge[thick] (s7)
        (s3) edge[thick] (lc4)
        (s4) edge[thick] (s8)
        (s5) edge[thick] (lc5)
        (s6) edge[thick] (s9)
        (s7) edge[thick] (lc6)
        (lc4) edge[thick] (s8)
        (s8) edge[thick] (lc5)
        (lc5) edge[thick] (s9)
        (s9) edge[thick] (lc6)
        ;
        
        
 \begin{pgfonlayer}{background}
        \node [background,
                    fit=(lc1) (lc6)] {};
    \end{pgfonlayer}
\end{tikzpicture}



%\caption{The general structure of an FPGA}
%\label{fig:fpga_structure}

